% Author: Adolfo Centeno 
% Kubeet Corp
% www.kubeet.com

 
\documentclass{beamer}
\setbeamertemplate{navigation symbols}{}
\usepackage[utf8]{inputenc}
\usepackage{beamerthemeshadow}
\usepackage{listings}
\usepackage{hyperref}


\begin{document}
\title{postgres - install}  
\author{Adolfo Centeno Tellez}
\date{\today} 

\begin{frame}
\titlepage
\end{frame}

\begin{frame}\frametitle{Table of contents}\tableofcontents
\end{frame} 




\section{Conceptos} 
\begin{frame}\frametitle{postgres} 

 Conceptos:
 
\begin{itemize}
\item install
\item connect/createdb    
\item create tables
\item insert some values
\item select values..
 

\end{itemize} 



\end{frame}

\subsection{install}

\begin{frame} 
Install from here..

\url{https://www.digitalocean.com/community/tutorials/como-instalar-y-utilizar-postgresql-en-ubuntu-16-04-es}


\end{frame}


\subsection{createdb / connect}

\defverbatim[colored]\lstcreatedb{
 \begin{lstlisting}[language=html,showstringspaces=false, basicstyle={\tiny}, keywordstyle=\color{red}]
 
 $ sudo -u postgres createdb shopcar
 $ sudo -u postgres createuser adsoft
 $ sudo -u postgres psql
 postgres=# alter user adsoft with encrypted password '5i5i5i5i';
 postgres=# grant all privileges on database shopcar to adsoft;
 
 \end{lstlisting}
}

\defverbatim[colored]\lstconnect{
 \begin{lstlisting}[language=html,showstringspaces=false, basicstyle={\tiny}, keywordstyle=\color{red}]

$ psql -d shopcar -U adsoft	
shopcar=# \list	
shopcar=# \conninfo 
 

 \end{lstlisting}
}

\begin{frame}


\begin{block}{create database}
  \lstcreatedb
\end{block}

\begin{block}{connect to postgres}
  \lstconnect
  
\end{block}


\end{frame}


\subsection{create tables}

\defverbatim[colored]\lstcreateprod{
 \begin{lstlisting}[language=html,showstringspaces=false, basicstyle={\tiny}, keywordstyle=\color{red}]
  shopcar=#  create table productos (id SERIAL NOT NULL, 
  codigo int, 
  nombre  varchar(100), 
  precio float, 
  exist int,
  "createdAt" date not null default CURRENT_DATE,
  "updatedAt" date not null default CURRENT_DATE);

 \end{lstlisting}
}


\defverbatim[colored]\lstprodtest{
 \begin{lstlisting}[language=html,showstringspaces=false, basicstyle={\tiny}, keywordstyle=\color{red}]
 shopcar=# insert into productos values(1, 'dron', 
           'dron with raspberry pi 3', 8000, 10);
 shopcar=# select * from productos;
 \end{lstlisting}
}


\defverbatim[colored]\lstprodpk{
 \begin{lstlisting}[language=html,showstringspaces=false, basicstyle={\tiny}, keywordstyle=\color{red}]
 shopcar=# alter table productos add constraint pk_productos
           primary key (codigo);
 \end{lstlisting}
}



\defverbatim[colored]\lstproddelete{
 \begin{lstlisting}[language=html,showstringspaces=false, basicstyle={\tiny}, keywordstyle=\color{red}] 
 shopcar=# delete from productos where codigo=4;
 shopcar=# delete from productos where precio > 5000 
           and exist > 10;
 shopcar=# delete from productos where precio > 5000 
           or nombre like '%dron%' 
 \end{lstlisting}
}



\defverbatim[colored]\lstprodupdate	{
 \begin{lstlisting}[language=html,showstringspaces=false, basicstyle={\tiny}, keywordstyle=\color{red}] 
 shopcar=# update productos set precio=1500 where codigo=3;
 shopcar=# update productos set precio=1000,
           exist=10, descripcion = 'intel edison' 
           where codigo=3; 
 \end{lstlisting}
}

\defverbatim[colored]\lstcreateprov	{
 \begin{lstlisting}[language=html,showstringspaces=false, basicstyle={\tiny}, keywordstyle=\color{red}] 
 shopcar=# create table proveedores(
   codigo int,
  nombre varchar(100),
 direccion varchar(200),
 lat float,
 lon float
 );
 
shopcar=# alter table proveedores add constraint
		 PK_prov primary key(codigo);
 \end{lstlisting}
}

\defverbatim[colored]\lstinsertprov	{
 \begin{lstlisting}[language=html,showstringspaces=false, basicstyle={\tiny}, keywordstyle=\color{red}] 
shopcar(# insert into proveedores values(1, 'intel',
 '5th street 13', 95.000, -200.234); 
shopcar=# insert into proveedores values(2, 'dell',
 '20th street 190', 120.23, -150.235);

 \end{lstlisting}
}

\defverbatim[colored]\lstcreateprodprov	{
 \begin{lstlisting}[language=html,showstringspaces=false, basicstyle={\tiny}, keywordstyle=\color{red}] 

shopcar=# create table prov_prod(
shopcar(# id_product int,
shopcar(# id_prov int,
shopcar(# precio float,
shopcar(# exist int
shopcar(# );

shopcar=# alter table prov_prod add constraint PK_prov_prod 
primary key (id_product, id_prov);

shopcar=# alter table prov_prod add constraint FK_id_product
foreign key (id_product) references productos(codigo);

shopcar=# alter table prov_prod add constraint FK_id_prov   
foreign key (id_prov) references proveedores(codigo);

\end{lstlisting}
}

\defverbatim[colored]\lstinsertprodprov	{
 \begin{lstlisting}[language=html,showstringspaces=false, basicstyle={\tiny}, keywordstyle=\color{red}] 


shopcar=# insert into prov_prod values (3, 1, 100, 10);
INSERT 0 1
shopcar=# insert into prov_prod values (3, 2, 100, 10);
INSERT 0 1
shopcar=# insert into prov_prod values (3, 1, 100, 10);

\end{lstlisting}
}



\defverbatim[colored]\lstsepomex	{
 \begin{lstlisting}[language=html,showstringspaces=false, basicstyle={\tiny}, keywordstyle=\color{red}] 

git clone https://github.com/redrbrt/sepomex-zip-codes.git
cd sepomex-zip-codes




\end{lstlisting}
}

\defverbatim[colored]\lstcreatesepomex	{
 \begin{lstlisting}[language=html,showstringspaces=false, basicstyle={\tiny}, keywordstyle=\color{red}] 

shopcar=# CREATE TABLE sepomex (
id  SERIAL NOT NULL,
idEstado INT  NOT NULL ,
estado VARCHAR(100) NOT NULL ,
idMunicipio SMALLINT NOT NULL ,
municipio VARCHAR(160) NOT NULL ,
ciudad VARCHAR(160),
zona VARCHAR(15) NOT NULL,
cp INT NOT NULL ,
asentamiento VARCHAR(700) NOT NULL ,
tipo VARCHAR(200) NOT NULL ,
PRIMARY KEY (id)
);

shopcar=#\i sepomex_abril-2016.sql


\end{lstlisting}
}


\begin{frame} 


\begin{block}{create table productos}
  \lstcreateprod
\end{block}


\begin{block}{testing ...}
  \lstprodtest
\end{block}

\begin{block}{creating Primary Key (PK) }
  \lstprodpk
\end{block}


\end{frame}


\begin{frame} 

\begin{block}{delete .. }
  \lstproddelete
\end{block}


\begin{block}{update .. }
  \lstprodupdate
\end{block}


\end{frame}


\begin{frame} 

\begin{block}{create proveedores .. }
 \lstcreateprov
\end{block}


\begin{block}{insert proveedores .. }
  \lstinsertprov
\end{block}


\end{frame}



\begin{frame} 

\begin{block}{create provprod}
 \lstcreateprodprov
\end{block}

\end{frame}



\begin{frame} 

\begin{block}{insert provprod}
 \lstinsertprodprov
\end{block}

\end{frame}



\begin{frame} 

\begin{block}{insert provprod}
 \lstsepomex
\end{block}

\end{frame}


\begin{frame} 

NOTE: edit sepomex-abril-2016.sql, delete create database, and create table statements

\begin{block}{create sepomex table}
 \lstcreatesepomex
\end{block}

\end{frame}

\end{document}
